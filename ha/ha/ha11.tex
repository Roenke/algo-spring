\section*{Снова потоки}
\begin{enumerate}
	\item Разберитесь с пересчётом потенциалов при поиске потока.
	
	\item Дан массив, найдите $k$ непересекающихся возрастающих последовательностей максимальной длины за $O(kV^2)$.
	
	\textbf{Решение.}
	
	Построим сеть: каждому элементу массива $a[1..n]$ сопоставим вершину $v_i$. Рассмотрим элемент массива с 
	индексом $i$. Рассмотрим элементы с индексами $j = i + 1, i + 2, ..., n$. Если $a[i] < a[j]$, то добавим ребро 
	$v_i \to v_j$. Стоимость ребра установим в $-1$, а пропускную способность равную $1$. 
	
	Теперь добавим исток и сток. Добавим вершину $s = v_0$ - исток, соединим её со всеми $v_i, i = 1,..,n$ рёбрами 
	$v_0 \to v_i$ стоимости 0, пропускной способности $1$. Так же добавим вершину $t = v_{n + 1}$ - сток, добавим 
	ребра $v_i \to v_{n + 1}, i = 1,..,n$ стоимости $0$, пропускной способности $1$. 
	
	Заметим, что все ребра в графе идут от вершин с меньшими номерами в вершины с большими номерами, значит циклов 
	отрицательного веса быть не может. Теперь, воспользовавшись результатом задачи $5$ с практики получим поток 
	размера $k$. 
	
	Заметим, что на каждом шаге поток имеет минимальную стоимость, а учитывая способ построения сети, 
	соответствующие подпоследовательности имеют максимальную суммарную длину.
	
	\item Дан граф, на каждом ребре написано $2$ числа $L$ и $R$ и $c$. По каждому ребру может течь не более
	чем $R$, но не менее, чем $L$ жидкости. Найдите:
	\begin{enumerate}
		\item произвольную циркуляцию
		\item произвольный поток
		\item максимальный поток
		\item поток минимальной стоимости.
	\end{enumerate}
	
	\item Есть k одинаковых автоматов и n заданий. Про каждое задание известно, во сколько его нужно начать делать, во сколько закончить, а также его стоимость. Каждый автомат может выполнять только одно задание в каждый момент времени. Нужно выполнить задания максимальной суммарной стоимости. $O(kn \log n)$.
\end{enumerate}
