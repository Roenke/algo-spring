\section*{Линейное программирование}
\begin{enumerate}
	\item Сведите к задаче линейного программирования задачу:
	\begin{equation*}
		\min\limits_{1 \leqslant i \leqslant p} \left[ \sum\limits_{j = 1}^n c_{ij} x_j \right] \to \max
	\end{equation*}
	
	При ограничениях 
	\begin{align*}
		& \sum\limits_{j = 1}^n a_{ij} x_j = b_i, i \in [1..m] \\
		& x_i \geqslant 0, i \in [1..n]
	\end{align*}
	
	\item Сведите к задаче линейного программирования задачу:
	\begin{equation*}
	\sum\limits_{i = 1}^p \left|\sum\limits_{j = 1}^n c_{ij}x_j - d_i\right| \to \min
	\end{equation*}
	
	При ограничениях 
	\begin{align*}
	& \sum\limits_{j = 1}^n a_{ij} x_j = b_i, i \in [1..m] \\
	& x_i \geqslant 0, i \in [1..n]
	\end{align*}
	
	\item Приведите пример несовместной задачи линейного программирования, двойственная к которой так же 
	несовместна.
	
	\item Докажите, что если политоп линейной программы в канонической форме $(Ax = b, x \geqslant 0)$ целый для 
	любого вектора $b$, то матрица $A$ тотально унимодулярна.
	
	\item Вспомним линейную программу для паросочетания в двудольном графе. Докажите, что для любого 
	недвудольного графа политоп заданный данной программой не будет целым.
	
	\item[*6.] Пусть у нас задан орграф $G = (V, E)$ с двумя выделенными различными вершинами $s, t \in V$ , для 
	каждого ребра $e$ которого задано вещественное неотрицательное число $c_e$ — его пропускная способность.
	
	$s-t$ потоком называется функция $f : E \to R^+$ такая, что:
	\begin{align*}
		\forall e \in E&: 0 \leqslant f_e \leqslant c_e \\
		\forall v \in V \setminus \{s, t\} &: \sum\limits_{e = (u,v)} f_e - \sum\limits_{e = (v,u)} f_e = 0
	\end{align*}
	
	Величиной потока называется:
	\begin{equation*}
		|f| = \sum\limits_{e = (s, u)} f_e - \sum\limits_{e = (u, s)} f_e
	\end{equation*}
	
	$s-t$ разрезом называется пара $(S, T = V \setminus S)$ подмножеств $V$ такая, что $s \in S$ и $t \in T$. 
	
	Весом разреза называется величина:
	\begin{equation*}
		\sum\limits_{e = u\in S, v\in T)} c_e
	\end{equation*}
	
	\begin{enumerate}
		\item Сведите задачу нахождения максимального $s-t$ потока к задаче линейного программирования.
		\item Пусть теперь граф $G$ будет \textbf{DAG}-ом, причем любая его вершина лежит на каком-то пути из $s$ 
		в $t$. Сведите задачу о минимальном $s-t$ разрезе к задаче линейного программирования.
		\item Сведите задачу о минимальном $s-t$ разрезе к задаче линейного программирования и покажите, что она 
		дуальна задаче о максимальном потоке, для произвольного неориентированного графа. Считайте, что 
		неориентированный граф это ориентированный граф, у которого для каждого ребра есть такое же в обратную 
		сторону.
		\item В произвольном орграфе сведите задачу о минимальном $s-t$ разрезе к задаче линейного 
		программирования и покажите, что она дуальна задаче о максимальном потоке.
	\end{enumerate}
\end{enumerate}



