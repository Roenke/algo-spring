\section{Деревья и пути}
\begin{enumerate}
	\item Пазовем ребро дерева тяжёлым, если размер поддерева, его нижнего конца больше либо равен половине 
	размера поддерева его верхнего конца. Покроем дерево путями следующим способом: из каждой вершины, в которую 
	входит тяжёлое ребро будем идти вверх, пока не дойдём до корня, или пока не пройдём по лёгкому ребру. Такое 
	покрытие называется \textbf{«Heavy-Light-Decomposition»}. Докажите, что путь из любой вершины в корень 
	пересекает не более чем $\log n$ путей из покрытия, где $n$ — количество вершин.
	
	\textbf{Решение.} 
	
	Рассмотрим путь от произвольной вершины до корня. Будем двигаться по нему в обратном направлении - от корня к вершине. При переходе к следующей вершине возможны два случая: прошли по тяжёлому ребру $\Rightarrow$ остались в прежнем пути, либо прошли по лёгкому ребру, значит начался другой путь. Заметим, что когда проходим по лёгкому ребру, то размер поддерева, в котором находится вершина уменьшился как минимум в 2 раза (иначе это ребро было бы тяжёлым). Таким образом, смена пути покрытия в пути от корня к произвольной вершине влечёт за собой уменьшение не менее чем в 2 раза размер поддерева, значит и путей покрытия, которые пересеклись с рассмотренным путём не более чем $\log_2 n$.
	
	\item Дано дерево, у каждой вершины есть вес. Запросы: 
	изменить вес вершины; 
	найти максимум на пути из $u$ в $v$. $\langle O(n \log n)$, $O(\log^2 n) \rangle$. 
	
	Hint: предыдущая задача.
	
	\textbf{Решение.} Построим \textbf{«Heavy-Light-Decomposition»}, и на каждом пути в этом разложении построим 
	дерево отрезков. Так же обеспечим возможность выполнять бинарный подъём на исходном дереве. После этого нужно 
	найти $LCA$ для $u$, $v$, так мы найдём путь из $u$ в $v$. Теперь найдём все пути в 
	Heavy-Light-Decomposition. И на каждом из них, с помощью дерева отрезков, найдём часть, которая пересекается 
	с путём из $u$ в $v$, и найдём на этой части максимум. При замене действия аналогичны: когда изменяем весь 
	вершины, то находим путь в декомпозиции и изменяем вес вершины в соответствующем дереве отрезков.
	
	Сложность описанного алгоритма $O(\log n)$ на поиск LCA, и $O(\log^2n)$ на поиск максимума на пути и на 
	изменение веса. Итоговая сложность $O(\log^2 n)$.
	\item Даны подвешенное дерево и его Эйлеров обход. Придумайте, как за $O(\log n)$ обновить Эйлеров обход при 
	переподвешивании дерева за другую вершину (для разных вариантов обхода).
	
	\textbf{Решение.} Вместе с Эйлеровым обходом будем хранить признак того, что вершина посещена в первый и 
	последний раз (например $v_b, v_e$ - для первого и последнего вхождения соответственно.). Тогда, чтобы 
	подвесить дерево за другую вершину $u$ достаточно из Эйлерова обхода вырезать строку  $V = u_b,..., u_e$. Эта 
	строка разбивает весь обход на 3 части : префикс $P$(всё что до $V$), $V$, и суффикс $S$(всё, что после $V$). 
	Затем нужно удалить первый символ $P$, получив $P'$, осталось получить результат в виде конкатенации 
	$VSP'$.
	
	Этот же алгоритм применим и в случае, когда вершина включается в Эйелров обход дважды: только при первом и 
	последнем посещении.
	
	Использование структуры данных \textit{rope} позволит выполнять требуемые операции над строчкой за $O(\log n)$
	
	\item Пусть есть много неподвешенных деревьев. Запросы \textit{online}:
	\begin{itemize}
		\item соединить ребром вершины $v$ и $u$ разных деревьев,
		\item удалить ребро между вершинами $v$ и $u$,
		\item проверить, в одной ли компоненте лежат вершины $u$ и $v$.
	\end{itemize}
	
	$O(n)$ на предобработку, $O(\log n)$ на запрос ($n$ — суммарный размер деревьев).
	
	\textbf{Решение.} 
	
	Заметим, что для подвешенных деревьев мы умеем решать эту задачу. В предыдущей задаче мы научились обновлять 
	за $O(\log n)$ Эйлеров обход дерева при переподвешивании за другую вершину.
	
	Предобработка: подвесим все имеющиеся деревья за произвольную вершину, найдём для них Эйлеровы обходы  с 
	построим Декартово дерево по неявному ключу. 
	
	Ответы на запросы:
	
	\textit{соединить ребром вершины $v$ и $u$ разных деревьев}. Переподвесить дерево, в котором находится 
	вершина $u$ за эту вершину. Соединить полученное подвешенное дерево с вершиной $v$.
	
	\textit{удалить ребро между вершинами $v$ и $u$}. Переподвесим дерево за $u$. Теперь $v$ является корнем 
	своего поддерева, задача сведена к уже решенной.
	
	\textit{проверить, в одной ли компоненте лежат вершины $u$ и $v$.} В любой момент времени все деревья 
	подвешены. Такая задача уже решена.
	
	\item Дано дерево из n вершин, по которому бегают муравьи. Вам поступает $m$ запросов вида $a, b$, что 
	означает, что очередной муравей пробежал из вершины $a$ в вершину $b$.
	
	\begin{enumerate}
		\item Выведите, сколько раз муравьи пробежали через каждое из ребер за $O(n + m)$.
		
		\textbf{Решение. } Заметим, что, если дерево подвешено, то с каждым ребром можно однозначно связать одну 
		вершину, которая является его нижним концом. Тогда можно считать, что мы прошли через ребро в том случае, 
		если посетили его нижний конец. Воспользуемся тем, что умеем искать LCA за $O(1)$. 
		
		Выполним следующие действия: с начала подвесим дерево за какую-то из вершин. Построим LCA (c $O(1)$) для 
		подвешенного дерева. Для запроса вида $a$, $b$ найдём LCA этих вершин - вершина $c$. К значению в 
		вершинах $a$, $b$ добавим единицу (т.к. ребра, которые заканчиваются в этой вершине посещаются по одному 
		разу), а от значения в вершине $c$ вычтем двойку(т.к. мы добавили две единицы к поддереву, но через 
		соответствующее ребро не ходили). После этого осталось обойти дерево в глубину и найти сумму значений в 
		вершинах поддерева для каждой вершины (т.к. если были пути в вершины поддерева, то они и проходили через 
		вершину - корень этого поддерева). По построению, полученное значение в каждой из вершин (кроме корня) 
		будет равно количеству походов через ребро, нижним концов которого является данная вершина.
		
		Сложность алгоритма равна сложности подвешивания дерева $O(n)$ + сложность построения LCA $O(n)$ + 
		обработка запросов $O(m)$ + $DFS$ $O(n)$. Итоговая сложность - $O(n + m)$.
	
		\item А теперь вам онлайн поступают вопросы про каждое из рёбер (эти запросы могут поступить в любой 
		момент, в том числе до окончания муравьиных пробегов). Обработайте каждый из двух типов запросов за 
		$log^2 n$ с предподсчетом за $O(n \log n)$.
		
		\textbf{Решение. } Эта задача в точности сводится к задаче номер два, только меняем значение не в одной 
		вершине, а на подотрезках, но в дереве отрезков мы умеем это делать. Любое ребро это путь длины один, 
		поэтому и запрос количестве проходов через ребро мы тоже умеем обработать. Предподсчет - построение 
		структуры дерева для бинарного подъёма, построение декомпозиции и дерева отрезков на каждом из путей 
		декомпозиции. Каждое их этих действий выполняется за $O(n \log n)$. 
	\end{enumerate}
\end{enumerate}
