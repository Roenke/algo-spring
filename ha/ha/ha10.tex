\section*{Паросочетания}
\begin{enumerate}
	\item Докажите, что в регулярном двудольном графе есть полное паросочетание.
	
	\textbf{Решение.} 
	
	Для начала, заметим, что регулярный двудольный граф -- граф у которого степени всех вершин совпадают. Обозначим 
	это число $k$. Из этого следует, что в графе количество вершин в каждой доле совпадает(Т.к. сколько рёбер вышло 
	из первой доли, столько же должно войти во вторую долю, а раз степени всех вершин равны, то и количество 
	вершин во второй доле такое же как и в первой.).
	
	Выберем произвольное подмножество вершин одной доли. Обозначим его $T$. Из него выходит $k \cdot |T|$ рёбер.
	Эти ребра приходят в множество $N(T)$. Заметим, что $\forall T: |T|\leqslant |N(T)|$. Почему это так. 
	Предположим противное, и $|N(T)| < |T|$. Но в $N(T)$ входит $k \dot |T|$ рёбер. Но тогда, по принципу Дирихле 
	найдётся вершина, в которую входит более чем $k$ рёбер из множества $T$. Но такого быть не может -- наш 
	граф регулярный. Получили противоречие. 
	
	Заметим, что мы доказали утверждение для любого $T \subseteq X$, значит выполнены условия теоремы Холла, 
	значит в графе существует $X-$насыщенное паросочетание. Но т.к. количество вершин в обоих долях графа 
	совпадает, то это паросочетание является совершенным.
	
	\item Для заданного клетчатого поля с дырками выберите максимальное количество попарно не смежных клеток. 
	Смежными считаются клетки с общей стороной.
	
	\textbf{Решение.}
	
	Построим граф $G$. Каждой клетке будет соответствовать вершина, а ребро между двумя вершинами проведём лишь в 
	том случае, если соответствующие клетки смежные. Заметим теперь, что построенный граф $G$ является двудольным 
	- т.к. клетчатое поле может быть представлено в виде шахматной доски, но мы соединили лишь вершины разных 
	цветов, значит ребра соединяют только вершины из разных долей графа.
	
	Осталось в таком двудольном графе найти максимальное независимое множество. Воспользуемся теоремой Кёнига, 
	которая утверждает, что минимальное вершинное покрытие в двудольном графе совпадает по размеру с максимальным 
	паросочетанием. Так же нам известно, что в произвольном графе максимальное независимое множество вершин - это 
	дополнение к минимальному вершинному покрытию.
	
	Значит нам необходимо найти максимальное паросочетание, затем по этому паросочетанию построить 
	минимальное вершинное покрытие. Дополнение к минимальному вершинному покрытию образуют максимальное 
	независимое множество. Это и есть искомое подмножество.
	
	Чтобы найти минимальное вершинное покрытие можно воспользоваться теоремой Кёнига, и следующим из его 
	доказательства алгоритмом.
	
	\item Разбейте вершины ориентированного графа на циклы. Т.е. каждая вершина должна быть покрыта ровно одним 
	циклом. Либо скажите, что это невозможно.
	
	\textbf{Решение.} Допустим, что граф можно разбить на циклы, тогда рассмотрим какой-то цикл $v_1, v_2, .., 
	v_n, v_1$. В нем есть ребра $v_1 \to v_2, v_2 \to v_3, v_3 \to v_4,..., v_n \to v_1 $. Заметим, что если 
	рассмотреть двудольный граф $G$, в котором первая доля - начала ребер, а вторая доля - концы, то цикл 
	соответствует некоторому совершенному паросочетанию в таком графе. Если же циклов несколько, то 
	соответствующие двудольные графы можно слить в один большой двудольный граф. В нем будет совершенное 
	паросочетание.
	
	Теперь обратно. Пусть в двудольном графе $G = [\{v_1, v_2, ..., v_n\}, \{v'_1, v'_2, ..., v'_n\}]$ существует 
	совершенное паросочетание $M$, тогда соответствующий орграф может быть разбить на циклы. Рассмотрим 
	совершенное паросочетание. Построим цикл, покрывающий $v_1$. Пусть в $M$ вершину $v_1$ покрывает ребро $v_1 
	\to v'_j$, тогда первое ребро искомого цикла будет $v_1 \to v_j$. Второе ребро будет $v_j \to v_i$, где $v_i$ 
	это вершина, которую покрывает ребро $(v_j \to v'_i) \in M$. Продолжая, мы обязательно получим замкнутый 
	цикл, т.к. вершина $v'_1$ тоже покрыта $M$. После этого, возможно покрыты все вершины, а если нет, то 
	аналогично покроем и их.
	
	Таким образом, доказана эквивалентность существования совершенного паросочетания и возможности разбить орграф 
	на циклы, в которых каждая вершина покрыта одним циклом. Способ построения циклов описан в доказательстве 
	обратного утверждения.
	
	\item Дано $N$ различных прямых. Нужно выбрать максимальное по размеру подмножество прямых такое, что никакие 
	две прямые не параллельны, и никакие прямые не пересекаются в точке c $x = 0$.
	
	\textbf{Решение.}
	
	Заметим, что каждую прямую можно характеризовать двумя значениями: угол наклона касательной, и точкой 
	пересечения с $Oy$ (возможно отсутствующей.). Построим следующий двудольный граф $G$. В первой доли находятся 
	все значения углов наклона, во второй доле находятся все значения точек пересечения с $Oy$. Добавим в граф 
	$N$ рёбер, которые соответствуют прямым. Если в полученном двудольном графе найти максимальное паросочетание, 
	то получим ровно то, что требуется, т.к. все соответствующие прямые имеют попарно различные точки пересечения 
	с $Oy$ и углы наклона касательной. 
	 
\end{enumerate}
