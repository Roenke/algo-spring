\section*{Паросочетания}
\begin{enumerate}
	\item Докажите, что в регулярном двудольном графе есть полное паросочетание.
	
	\textbf{Решение.} 
	
	Для начала, заметим, что регулярный двудольный граф -- граф у которого степени всех вершин совпадают. Обозначим 
	это число $k$. Из этого следует, что в графе количество вершин в каждой доле совпадает(Т.к. сколько рёбер вышло 
	из первой доли, столько же должно войти во вторую долю, а раз степени всех вершин равны, то и количество 
	вершин во второй доле такое же как и в первой.).
	
	Выберем произвольное подмножество вершин одной доли. Обозначим его $T$. Из него выходит $k \cdot |T|$ рёбер.
	Эти ребра приходят в множество $N(T)$. Заметим, что $\forall T: |T|\leqslant |N(T)|$. Почему это так. 
	Предположим противное, и $|N(T)| < |T|$. Но в $N(T)$ входит $k \dot |T|$ рёбер. Но тогда, по принципу Дирихле 
	найдётся вершина, в которую входит более чем $k$ рёбер из множества $T$. Но такого быть не может -- наш 
	граф регулярный. Получили противоречие. 
	
	Заметим, что мы доказали утверждение для любого $T \subseteq X$, значит выполнены условия теоремы Холла, 
	значит в графе существует $X-$насыщенное паросочетание. Но т.к. количество вершин в обоих долях графа 
	совпадает, то это паросочетание является совершенным.
	
	\item Для заданного клетчатого поля с дырками выберите максимальное количество попарно не смежных клеток. 
	Смежными считаются клетки с общей стороной.
	\item Разбейте вершины ориентированного графа на циклы. Т.е. каждая вершина должна быть покрыта ровно одним 
	циклом. Либо скажите, что это невозможно.
	\item Дано $N$ различных прямых. Нужно выбрать максимальное по размеру подмножество прямых такое, что никакие 
	две прямые не параллельны, и никакие прямые не пересекаются в точке c $x = 0$.
	
	\textbf{Решение.}
	
	Заметим, что ограничения данные на прямые, которые нельзя одновременно включить в искомое подмножество, не 
	есть отношение эквивалентности(отсутствует транзитивность). Это не позволяет нам разбить множество прямых 
	на классы эквивалентности и взять по представителю из каждого класса, получив искомое подмножество. 
	
	Перенумеруем прямые. Построим двудольный граф $G[X,Y]$ из $2N$ вершин, по $N$ в каждой доле. Занумеруем и 
	их - в соответствии с нумерацией прямых. Проведём между вершинами $v_{i} \in X, u_j \in Y$ в том случае, 
	если их нельзя включить в в искомое подмножество $D$. Это всё мы успеем выполнить за $O(N ^ 2)$, т.к. рёбер 
	в таком графе не более $N^2$, а проверка - нужно ли проводить ребра выполняется за $O(1)$. 
	
	Теперь у нас есть двудольный граф. Заметим, что нам нужно найти в нём максимальное независимое множество. 
	Но мы знаем, что это $NP-$трудная задача в общем случае. Но ведь наш граф двудольный, и это несколько 
	упрощает задачу. Вспомним, что дополнение вершинного покрытия есть независимое множество, и наоборот. 
	Значит, максимальное независимое множество, это дополнение к минимальному вершинному покрытию. По теореме 
	Кенига мы умеет искать минимальное вершинное покрытие по имеющемуся максимальному паросочетанию. Таким 
	образом, задача сведена к поиску максимального паросочетания в двудольном графе. С этой задачей справится 
	алгоритм Куна за $O(N^3)$.
	
	Таким образом, итоговая сложность алгоритма составляет $O(N^3)$.
	 
\end{enumerate}

\subsection*{Дополнительные задачи}
\begin{enumerate}
	\item Приведите полиномиальный алгоритм для нахождения чётности количества совершенных паросочетаний в 
	двудольном графе с равным размером долей.
\end{enumerate}