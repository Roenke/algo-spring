\section*{Алгебра}
\begin{enumerate}
	\item Дана матрица $A$ размера $n \times m$. Найти такое представление $A = B \cdot C$, что $B$ имеет размер $n 
	\times k$ и $k$ минимально.
	
	\item Для заданного $n$ и простого $p$ найти число таких $k \in [0, n]$, что $\binom{n}{k} \mod p = 0$. Можно 
	считать, что все операции по модулю $p$ происходят за $O(1)$. Придумайте алгоритм, который работает за:
	\begin{enumerate}
		\item $O(n)$
		\item $O(loly(\log n))$ (Подсказка: попробуйте представить числа в $p$-ичной системе счисления и упростить $(x + 1)^n$)
	\end{enumerate}
	
	\item На кольцевой стоят $n$ светофоров и хитро перемигиваются. Свет на каждом светофоре зажигается в таком 
	порядке: зелёный-жёлтый-красный, снова зелёный и так по кругу. Все светофоры пронумерованы от $1$ до $n$. Если 
	светофор по номеру $i$ решит поменять свой цвет то следующие, $k_i$ светофоров с шагом $a_i$ тоже поменяют свой 
	цвет. Ниже приведён пример, когда переключается	первый светофор: $k_1 = 6, a_1 = 2$.
	
	По начальному состоянию светофоров определите, возможен ли зелёный коридор. Т.е. такая ситуация, когда все 
	светофоры переключены на зелёный. Решить за $O(n^3)$.

	\item У Винни-Пуха есть бесконечные запасы из $k$ видов горшочков мёда. Каждый вид горшочков характеризуется 
	целым числом — сколько дней нужно Винни-Пуху, чтобы съесть весь мёд из одного такого горшка. Винни-Пух уже 
	провёл серию экспериментов вида "взять несколько горшочков мёда, записать день недели, когда эксперимент 
	начался, есть мёд, пока тот не закончится, записать день недели, когда эксперимент закончился". По понятным 
	причинам Винни очень любит экспериментировать. А Кролик не любит мёд, но любит предсказывать будущее. Помогите 
	Кролику, зная результаты первых $k$ экспериментов, сказать, можно ли однозначно сказать результат следующего 
	эксперимента. Оцените время работы алгоритма.
	\begin{enumerate}
		\item Винни берёт произвольные множества горшков. $O(k^3)$.
		\item Каждый раз Винни берёт горшки не более чем двух разных типов. $O(k)$.
	\end{enumerate}
	
\end{enumerate}
	
\subsection{Дополнительная задача}
\begin{enumerate}
	\item Рассмотрим матрицу $A \in \{ 0, 1 \}^{n \times m }$. Для произвольных $i$ и $j$ $(1 \leqslant i 
	\leqslant n, 1 \leqslant j \leqslant m)$ разрешается поменять все значения в строке $i$ и столбце $j$ на 
	противоположные (значение на пересечении строки и столбца меняется). Требуется получить нулевую или единичную 
	матрицу. Придумайте алгоритм за $o((nm)^3)$.
\end{enumerate}



