\section*{Алгебра}
\begin{enumerate}
	\item Дана матрица $A$ размера $n \times m$. Найти такое представление $A = B \cdot C$, что $B$ имеет размер $n 
	\times k$ и $k$ минимально.
	
	\textbf{Решение.} Искомое разложение можно получить, воспользовавшись
	 \url{https://en.wikipedia.org/wiki/Rank_factorization}.
	 
	 Осталось понять, почему в таком разложении $k$ минимально. Для ранга произведения двух матриц справедливо следующее утверждение (общий факт из линейной алгебры):
	 \begin{equation*}
	 	rank(A) = rank(BC) \leqslant \min(rank(B), rank(C))
	 \end{equation*}
	 
	 Из неравенства видно, что взять $k$ меньше чем ранг невозможно. В нашем разложении $k = rank(A) = rank(B) = 
	 rank(C)$
	
	\item Для заданного $n$ и простого $p$ найти число таких $k \in [0, n]$, что $\binom{n}{k} \mod p = 0$. Можно 
	считать, что все операции по модулю $p$ происходят за $O(1)$. Придумайте алгоритм, который работает за:
	\begin{enumerate}
		\item $O(n)$
		
		\textbf{Решение.} Вспомним формулу:
		\begin{equation*}
		\binom{n}{k} = \frac{n!}{k!(n - k)!}
		\end{equation*}
		
		Заметим, что это значение может быть представлено в виде $\binom{n}{k} = p^m \cdot r$. Где $r$- число, 
		которое не делится на $p$, а $m$ - некоторое неотрицательное число (бин. коэффициент не может быть 
		дробным).
		
		Осталось научиться быстро искать $m$ для разных значений $k$.
		
		Можно воспользоваться следующей динамикой для предподсчета степеней $p$, которые входят в $1!, 2!, .., 
		n!$ и в значения $1,2,3, .., n$. Массив для первой динамики обозначим $d$, для второй $r$.
		\begin{align*}
			&r[1, ..., p - 1] = 0 \\
			&r[i] = 0, i \neq 0 (mod \ p) \\
			&r(i) = r(i / p) + 1, i = 0(mod \ p)
		\end{align*}
		
		И вторая:
		\begin{align*}
		&d[1] = d[2..p-1] = 0 \\
		&d[i] = d[i - 1], i \neq 0 (mod \ p) \\
		&d[i] = d[i - 1] + r[i], i = 0(mod \ p)
		\end{align*}
		
		Теперь, заметим, что $m = d[n] - d[k] - d[n - k]$. Это можно вычислить за $O(1)$, осталось лишь найти количество таких $k$ из интервала $[1, n]$. 
		
		\textbf{Сложность:} $O(n)$ на каждую из динамик $O(n)$ на поиск всех $k$. Итого $O(n)$.
		
		\item $O(loly(\log n))$ (Подсказка: попробуйте представить числа в $p$-ичной системе счисления и 
		упростить $(x + 1)^n$)
		
		Можем воспользоваться теоремой Лукаса \url{https://en.wikipedia.org/wiki/Lucas'_theorem}, тогда
		\begin{equation*}
		\binom{n}{k} = \prod_{i=0}^d\binom{n_i}{k_i}\pmod p,
		\end{equation*}
		Где $n_i, k_i$ - разряды разложения чисел $n,k$ в $p$-ичную систему исчисления. Такие разложения можем 
		найти за $O(\log n)$.
		
		Тогда, чтобы произведение было равно $0$, необходимо, чтобы хотя бы один множитель был равен $0$. Чтобы 
		проверить, множитель равен $0$ достаточно сравнить $n_i$ с $k_i$, и если $k_i > n_i$, то коэффициент 
		равен 0, и ответ на вопрос задачи положительный. А т.к. количество слагаемых не превосходит $O(\log n)$, 
		то и все сравнения мы успеем сделать за $O(\log n)$.
		
	\end{enumerate}
	
	\item На кольцевой стоят $n$ светофоров и хитро перемигиваются. Свет на каждом светофоре зажигается в таком 
	порядке: зелёный-жёлтый-красный, снова зелёный и так по кругу. Все светофоры пронумерованы от $1$ до $n$. Если 
	светофор по номеру $i$ решит поменять свой цвет то следующие, $k_i$ светофоров с шагом $a_i$ тоже поменяют свой 
	цвет.
	
	По начальному состоянию светофоров определите, возможен ли зелёный коридор. Т.е. такая ситуация, когда все 
	светофоры переключены на зелёный. Решить за $O(n^3)$.
	
	\textbf{Решение.}
	
	Для начала скажем, что значения $a_i$ можно взять по модулю $n$, т.к. у нас светофоры стоят в кругу, и 
	наматывать круги большого смысла нет. Так же заметим, что после $lca(a_i, n)$ шагов мы точно начнём ходить по 
	тем же светофорам, а после $3*lca(a_i, n)$ и вовсе делать то же самое. Поэтому, можно считать, что мы то же 
	самое не делаем, и все $k_i$ ограничены $n^2$. 
	
	Теперь заполним матрицу $L_{n\times n}$, где $l_{ij}$ - количество переходов светофора с номером $j$, после 
	того, как светофор $i$ сменил свой цвет. За $O(n^3)$ мы это успеем сделать, т.к. каждую строку мы заполняем 
	не более чем за $O(n^2)$ переходов. 
	
	Очевидно, что порядок включений светофоров не важен. Тогда попытаемся понять, как же определить какого цвета 
	станет светофор с номером $i$ после каких-то переключений светофоров, где светофор номер $1$ переключится 
	$k_1$ раз, второй $k_2$ раз, и т.д. до $k_n$. Тогда цвет светофора номер $j$ определяется его начальным 
	состоянием($state_j$) и переходами, которые были вызваны переключениями других светофоров. Будем считать, что 
	зелёному соответствует $0$, жёлтому $1$, красному $2$.
	\begin{equation*}
		s_j = state_j + \sum\limits_{i = 1}^{n} L[i][j] * k_i = 0
	\end{equation*}
	
	Несложно заметить, что записав такие равенства для всех светофоров мы получим СЛАУ с матрицей $L^T$, вектором 
	неизвестных из $k_i$ и вектором правой части из начальных состояний. Попытаемся её решить (по простому модулю 
	3), Если получится, то последовательность переключений будет решением, если решения нет, то и 
	последовательности тоже нет.
	 
	\item У Винни-Пуха есть бесконечные запасы из $k$ видов горшочков мёда. Каждый вид горшочков характеризуется 
	целым числом — сколько дней нужно Винни-Пуху, чтобы съесть весь мёд из одного такого горшка. Винни-Пух уже 
	провёл серию экспериментов вида "взять несколько горшочков мёда, записать день недели, когда эксперимент 
	начался, есть мёд, пока тот не закончится, записать день недели, когда эксперимент закончился". По понятным 
	причинам Винни очень любит экспериментировать. А Кролик не любит мёд, но любит предсказывать будущее. 
	Помогите Кролику, зная результаты первых $k$ экспериментов, сказать, можно ли однозначно сказать результат 
	следующего эксперимента. Оцените время работы алгоритма.
	\begin{enumerate}
		\item Винни берёт произвольные множества горшков. $O(k^3)$.
		
		\textbf{Решение}. 
		
		Будем считать, что медведь использует в каждом эксперименте не более 1 горшка мёда каждого вида. Тогда 
		изначально нам даны битовые маски длины $k$ и некоторое значение по модулю 7. Так же даётся ещё одна 
		строка - следующий эксперимент. Задача сводится к задаче нахождения разложения новой битовой строки по 
		$k$ имеющимся. И решается и помощью метода Гаусса. Если получилось, то можем предсказать, иначе - не 
		можем. 
		\item Каждый раз Винни берёт горшки не более чем двух разных типов. $O(k)$.
		
		В этом случае все битовые маски длины не более 2. Либо и вовсе 1. Когда 1, то тогда мы всё знаем о мёде этого вида. В случае, когда 2, то мы можем предсказать результат эксперимента лишь в том случае, когда такой же эксперимент уже был проведён до этого, либо когда мы знаем по отдельности результаты экспериментов для каждого из сортов мёда, которые выбрал медведь в своём последнем эксперименте.
	\end{enumerate}
	
\end{enumerate}



