\section*{RSA и FFT}
\begin{enumerate}
	\item Алена отправила сообщение $m$, зашифрованное через \textbf{RSA}, двум людям. Для каждого человека 
	определено свое $e_i$, но везде одинаковое $n = pq$. Оказалось, что $e_i$ взаимно простые. Найдите сообщение 
	Алены за $O(poly(\log n)$.
	
	\textbf{Решение.} 
	
	Заметим, что с помощью расширенного алгоритма Евклида можно найти $u, v \in \mathbb{Z}$ такие, что $ue_1 + 
	ve_2 = 1$.
	
	Обозначим отправленные шифры $c_1 = m^{e_1}$, $c_2 = m^{e_2}$. Заметим, что в этом случае $c_1^u c_2^v = 
	(m^{e_1})^u\cdot(m^{e_2})^v = m^{ue_1 + ve_2 = 1}$. Но выражение в степени равно единице, таким образом: 
	$c_1^u c_2^v = m$. То есть мы научились искать $m$.
	
 	\textbf{Сложность.} Если считать, что операции в кольце по модулю, $n$ выполняются за $O(1)$, а алгоритм 
 	Евклида работает за $O(\log n)$. То получим, требуемую сложность.
	
	\item 
	\begin{enumerate}
		\item Пусть $n = pq$ и известно $\varphi(n)$, разложите $n$ на множители. $O(poly(\log n))$.
		
		\textbf{Решение.} Посмотрим на то, что нам известно:
		\begin{align*}
			n &= pq \\
			\varphi(n) &= (p - 1) * (q - 1)
		\end{align*} 
		
		Заметим, что это невырожденная нелинейная система, с двумя уравнениями и двумя неизвестными $p, q$, решив которую получим ответ к задаче. 
			
		\item Пусть вам дано \textbf{RSA}$(n = pq, e, d)$. Пусть $e = 3$. Разложите $n$ на множители.
	\end{enumerate}
	
	\item[4.] Придумайте, как свести вычисление \textbf{FFT} последовательности размера $pn$ к $p$ вычислениям 
	\textbf{FFT} от	последовательностей размера $n$ и $O(p^2n)$ дополнительных арифметических операций. Напишите 
	псевдокод.
	
	\textbf{Решение.}
	
	\url{https://en.wikipedia.org/wiki/Cooley%E2%80%93Tukey_FFT_algorithm#General_factorizations}
	
	\item[6.] Заданы картинка $a$ и образец $p$ в виде матриц вещественных чисел из $[0, 1]$ размерами $n \times n$ 
	и $k \times k$ соответственно $(n \geqslant k)$. Требуется найти позицию $(x, y), 0 \leqslant x \leqslant n - 
	k, 0 \leqslant y \leqslant n - k$, для которой:
	\begin{equation*}
		\sum\limits_{i = 0}^{k - 1} \sum\limits_{j = 0}^{k - 1} (p_{i,j} - a_{(y + i), (x + j)})^2 \rightarrow min
	\end{equation*}
	за время $O(n^2\log n)$
	
	\textbf{Решение.}
	
	\item[7.] Пусть вам известно \textbf{FFT}$([a_1, a_2, \cdots , a_n])$. Придумайте, как пoсчитать 
	\textbf{FFT}$([a_2, a_3, \cdots , a_{n+1}])$ за линию.
	
	\textbf{Решение.} Подумаем, что же нам даёт \textbf{FFT}$([a_1, a_2, \cdots , a_n])$. А дает он нам значения 
	многочлена $A(x) = a_n x^{n - 1} + a_{n - 1} x^{n - 1} + \dots + a_2 x + a_1$ в $n$ точках $x_1, x_2, \dots, 
	x_n$, то есть набор $A(x_1), A(x_2), \dots, A(x_n)$.
	
	Теперь рассмотрим, что мы хотим узнать в \textbf{FFT}$([a_2, a_3, \cdots , a_{n+1}])$. Это значения другого 
	многочлена $B(x) = a_{n + 1} x_{n - 1} + a_n x_{n - 2} + \dots a_3 x + a_2$ в каких-то $n$ точках. 
	
	Оставим точки прежними, то есть  $x_1, x_2, \dots, x_n$, и заметим, что значения $B(x_i)$ можно быстро 
	пересчитать, зная значения $A(x_i)$, т.к. выражения, определяющие их, очень похожи. Так, получим:
	\begin{equation*}
		B(x_i) = \frac{A(x_i) - a_1}{x_i} + a_{n + 1} * x_i^{n - 1}
	\end{equation*}
	
	Если однажды закешировать значения $x_i^{n - 1}$, можно быстро пересчитывать все $B(x_i)$. 
	
	\textbf{Сложность.} Каждое $B(x_i)$ пересчитываем за $O(1)$, всего их $n$ штук, следовательно общая сложность 
	составит $O(n)$.
	
\end{enumerate}



