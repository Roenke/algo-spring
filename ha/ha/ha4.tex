\section*{Хеширование}
\begin{enumerate}
	\item Семейство хэш-функций $\mathcal{H} = \{h : X \rightarrow Y \}$ называется универсальным, если:
	
	\begin{equation*}
	\forall x_1, x_2 \in X, x_1 \neq x_2 : \underset{h \in \mathcal{H}}{\mathbf{Pr}} \left[h(x_1) = h(x_2)\right] \leqslant \frac{1}{|Y|}
	\end{equation*}
	
	Семейство хэш-функций $\mathcal{H} = \{h : X \rightarrow Y \}$ называется $k$-независимым, если для любых различных $x_1, x_2, \cdots , x_k \in X$, для любых, возможно совпадающих, выполняется: $y_1, y_2, \cdots, y_k \in Y$
	
	\begin{equation*}
	 \underset{h \in \mathcal{H}}{\mathbf{Pr}} \left[\bigwedge\limits_{i = 1}^{k} h(x_i) = y_i\right] = \frac{1}{|Y|^k}
	\end{equation*}
		
	\begin{enumerate}
		\item Докажите, что из любое 2-независимое семейство хэш-функций является универсальным.
		\item Докажите, что любое $k + 1$-независимое семейство хэш-функций является $k$-независимым.
	\end{enumerate}
	
	\item В вашем распоряжении есть пара 2-независимых семейств хеш-функций $\mathcal{A} = \{f : A \rightarrow F_2^n \}$ и $\mathcal{B} = \{g : B \rightarrow F_2^n \}$. Постройте (и докажите, что построено правильно) универсальное семейство хеш-функций, которое:
	\begin{enumerate}
		\item будет отправлять пары типа $(A, B)$ в $F_2^n$,
		\item будет отправлять мультимножества из элементов типа $A$ в $F_2^n$,
		\item будет отправлять множества из элементов типа $A$ в $F$,
		\item будет отправлять списки из элементов типа $A$ в $F_2^n$.
	\end{enumerate}
		
\end{enumerate}


